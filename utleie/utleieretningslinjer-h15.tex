\documentclass[norsk, 12pt]{article}
\usepackage[norsk]{babel} 
\usepackage[utf8]{inputenc} 
\usepackage{times}			% Default times font style
\usepackage[T1]{fontenc} 	% Font encoding
\usepackage{amsmath} 		% Math package

\usepackage{hyperref}
\usepackage[letterpaper, margin=1in]{geometry}
\usepackage{parskip}		% Norske avsnitt

\title{}

\begin{document}
\section*{Noen retningslinjer for utleiearrangementer}
Å jobbe på utleiearrangementer er litt annerledes enn på en vanlig fredagspub eller
en stor fest. Hovedforskjellen er at jobben er betalt, og derfor stilles det større krav 
ved profesjonaliteten til de som jobber. 
Dette betyr blant annet at det som oftest er færre ansatte per gjest 
enn vanlig, så arbeidsmengden er gjerne litt større, mens klientellet ofte er 
enklere å jobbe med (det finnes uhederlige unntak).

Ved utleiearrangementer hjelper alle hverandre, så vakt rydder tomme glass, flasker o.l.,
og bartendere har ansvar for ikke å skjenke gjester slik at de kan bli «åpenbart påvirket»,
og å sjekke legitimasjon.
Skjenkemestere har ansvaret for å starte/stenge tappeanlegget, ta oppgjør,
og at det ved utleiets begynnelse er nok drikkevarer satt kaldt
\footnote{Det gjelder
\textit{spesielt} alkoholfri øl og brus!}, 
samt at lokalet ser
skikkelig ut før og etter arrangementet. Det er ingen ekstra betaling for
denne jobben. Alle som jobber får betalt for totalt én time utover arrangementets
varighet for å rydde før/etter arrangementet (typisk en halvtime før/etter).
Oppmøte er derfor en halvtime før arrangementet/skjenking begynner!

Skjenking av alkohol kan ikke skje etter kl. 02.00, og skjenkingen må
senest opphøre en halvtime før stenging 
(f.eks. en gang mellom 01.30 og 02.00 hvis stengetid er 02.00).
Hvis det er avtalt med utleiesjef og skjenkemester kan det serveres brennevin,
men det er kun hvis det ikke er noen under 20 år i lokalet. 
Hvis det er under 50 gjester kan personalet slippe folk ut og inn av bygget
hvis utgangsdøren er stengt. Hvis det er over 50 gjester, må man ha en dørvakt.

Det er ikke jobbegoder ved utleiejobbing.
\footnote{Unntaket er kaffe, 
som man kan drikke gratis så lenge man husker å rense kaffekannene etter bruk.}
Det inkluderer ryddepizza/-bonger, 
mat/snacks underveis mm. De som jobber står likevel fritt til å kjøpe inn egen 
mat/snacks/drikke.
Det er som ved alle RF-arrangementer ikke lov å drikke alkohol mens man 
håndterer penger/alkohol/fulle mennesker.

Lønnen for å jobbe ved utleiearrangementer er 130,–/time, og ved jobbing over ni timer
øker lønnen med 50 \% per time etter dette, med mindre noe annet spesifiseres.
Hvis man jobber for RF for første gang, må man overlevere fullt navn, 
\textit{folkeregistrert} adresse, kontonummer og personnummer til RFs utleiesjef.

Arbeidsgiver og lønnsutbetaler er \textit{Studentenes personalforening (SPF)}, 
som alle foreninger som driver utleievirksomhet ved UiO er pliktige å være en del av.
Dvs. at SPF sender en faktura til RF for personellet ved utleiearrangementet.
Lønn utbetales som oftest innen den 20. hver måned, men det er ikke nødvendigvis slik.
En mulig grunn til at du ikke har mottatt lønn er at RF ikke har betalt fakturaen fra
SPF, eller at semesteret ikke har begynt ennå (sommer-/juleferie).
\end{document}