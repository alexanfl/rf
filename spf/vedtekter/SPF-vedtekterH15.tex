\documentclass[11pt]{article}
\usepackage[utf8]{inputenc}
\usepackage[norsk]{babel}
\usepackage{lmodern}
\usepackage[a4paper, margin=1in]{geometry}
\usepackage{lastpage}
\usepackage{fancyhdr}

\begin{document}
\pagestyle{fancy}
\fancyhf{}
\fancyhead[L]{\textit{Vedtekter for Studenkjellernes personalforening av 10.09.2015}}
\fancyhead[R]{\thepage\ av\ \pageref{LastPage}}
\title{Vedtekter for Studentkjellernes personalforening}
\date{av 10.09.2015}

\begin{center}
    {\LARGE\textbf{Vedtekter for Studenkjellernes personalforening}}\\[7pt]
    av 10.09.2015	
\end{center}


\section{Navn}
\label{sec:1}
Foreningens navn er \emph{Studentkjellernes personalforening}.
\section{Formål}
\label{sec:2}
Foreningens formål er å administrere personaltjenester for medlemmer av 
Studentkjellernes personalforening.
Med personaltjenester menes lønnet arbeid utført på arrangementer hos medlemsforeningene.
\section{Medlemskap}
\label{sec:3}
\subsection{}
\label{sub:3.1}
Alle foreninger som har egen bardrift ved Universitet i Oslo kan bli medlemmer.
\subsection{}
\label{sub:3.2}
Nye medlemmer tas opp ved skriftlig forespørsel til styret.
\subsection{}
\label{sub:3.3}
Alle medlemmer har en egen medlemsavtale som skal regulere ansvarsfordelingen
mellom Studentkjellernes personalforening og hver enkel medlemsforening.
Avtalen nyforhandles hvert høstsemester og skal være på plass senest
fire uker etter generalforsamling. 
Brudd på medlemsavtalen eller forsømmelse av avtaleforhandling
kan føre til suspensjon eller oppsigelse av medlemskapet.
\subsection{}
\label{sub:3.4}
Alle medlemsforeninger har et medlem og et varamedlem i styret.
Styrerepresetanten og varamedlemmet velges av medlemsforeningen.
\section{Generalforsamling}
\label{sec:4}
\subsection{}
\label{sub:4.1}
Generalforsamling holdes to ganger i året, i februar og i september.
Generalforsamling innkalles med tre ukers varsel. 
Generalforsamlingen skal behandle styrets beretning, regnskap og budsjett. 
Generalforsamlingen skal godkjenne de styremedlemmer medlemsforeningen 
har oppnevnt til styret, 
og velge leder for Studentkjellernes personalforening blant disse.
\subsection{}
\label{sub:4.2}
Ekstraordinær generalforsamling innkalles med to ukers varsel. 
Styret eller to av medlemsforeningene kan innkalle til ekstraordinær generalforsamling.
\subsection{}
\label{sub:4.3}
En representant for hver av medlemsforeningene har stemmerett på generalforsamlingen.
Styret i Studentkjellernes personalforening, styremedlemmer i medlemsforeningene
og de som har mottatt honorar fra Studentkjellernes personalforening har møterett på
generalforsamlingen.
\subsection{}
\label{sub:4.4}
Generalforsamlingen er stemmeberettiget når 3/4 av medlemmene er tilstede.
\section{Styret}
\label{sec:5}
\subsection{}
\label{sub:5.1}
Styrets hovedoppgave er å forvalte medlemsforeningenes behov for personaltjenester.
De skal ta opp nye medlemmer, inngå medlemsavtaler og følge disse.
\subsection{}
\label{sub:5.2}
Styret er sammensatt av en representant med vararepresentant fra hver medlemsforening.
\subsection{}
\label{sub:5.3}
Styreleder velges på generalforsamling og vervet har en varighet på et halvt år.
Styreleder velges av generalforsamlingen blant styremedlemmene.
Så langt det er mulig skal vervet som styreleder gå på omgang blant medlemsforeningene.
Styreleder er daglig leder for foreningen.
\subsection{}
\label{sub:5.4}
Styret skal i tillegg til styreleder ha en økonomiansvarlig, nestleder og sekretær.
Disse vervene har en varighet på et halvt år og velges av generalforsamlingen
blant styremedlemmene.
\subsection{}
\label{sub:5.5}
For å være valgbar til styret må en styrerepresentant være aktiv frivillig i sin egen
forening ved starten av vervet, og godkjent av sin egen forenings hovedstyre.
\subsection{}
\label{sub:5.6}
Hvis en styrerepresentant fratrer sitt verv før tiden skal medlemsforeningen
vedkommende kommer fra velge en ny representant innen en uke.
Dersom styrelederen trekker seg før valgperioden er ferdig,
skal det innkalles til ekstraordinær generalforsamling for
å velge en ny leder for den resterende delen av valgperioden.
\subsection{}
\label{sub:5.7}
Varamedlemmer har møterett i styret.
\section{Protokolltilførsel}
\label{sec:6}
Styret skal føre og godkjenne protokoll fra alle sine møter.
Kopi av protokoll skal sendes til driftslederne i alle medlemsforeningene.
\section{Økonomi}
\label{sec:7}
Foreningen skal ikke ha som mål å tjene penger og skal derfor ikke gå med overskudd.
Foreningen har mulighet til å opparbeide seg et fond.
\section{Signatur}
\label{sec:8}
Daglig leder har signaturrett.
\section{Vedtekter}
\label{sec:9}
Vedtektene kan endres på ordinær generalforsamling og ekstraordinær generalforsamling.
Vedtaket må være enstemmig. Medlemsforeningenes hovedstyrer er høringsorgan
for vedtektsendringer. Vedtektsendringer skal være klare og sendes ut til
medlemsforeningene senest to uker før generalforsamling.
\section{Oppløsning}
\label{sec:10}
Foreningen kan oppløses ved vedtak om oppløsning på to etterfølgende generalforsamlinger.
Vedtaket må være enstemmig. Foreningens gjenværende eiendeler skal forvaltes
av SiO ved Studentlivet.

\end{document}
