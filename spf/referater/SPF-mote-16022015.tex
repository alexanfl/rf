\documentclass[12pt,norsk]{article}
\usepackage[utf8]{inputenc}
\usepackage[norsk]{babel}
\usepackage{lmodern}
\renewcommand{\thesubsection}{\thesection \alph{subsection}}

\begin{document}

\title{SPF-møtereferat}
\author{Alexander Fleischer}
\maketitle



\section{Godkjenning av innkalling og referat}
Godkjenner innkalling.
Godkjenner referatet fra 3. februar.

\section{Orienteringer}
\subsection{Møte med SiO}
- Diskuterte ståa i SPF nå. Gamle utbetalinger/nye utbetalinger.
Vil separere kontoene i Norlønn, så det er én for gamle utbetalinger,
og én for 2015. Betaler ikke ut lønn før vi har fått inn penger fra foreningene.

Har fokus på å holde stålkontroll på 2015, slik at alt er greit der.
Går også gjennom 2014 for å lukke det regnskapet.
Skal ha et nytt møte med Mike/SiO. Vi får nok neppe
kontor av dem. 
Skal vi sette krav om at SPF-representanter
tar/har SiO-økonomikurs?

Lukker perioden den 15. hver måned. Fakturaene sendes ut den 20.
Viktig at foreningene betaler innen slutten av måneden,
ellers betales det ikke ut lønn før uken etter.

\subsection{Annet}
Christiane har ryddet i dokumentene som hun fikk fra Lars Christian.
Hun har også funnet vedtektene til SPF.
Brønnøysundregistreringen er godkjent.
Christiane har byttet program for regnskapet. Laster opp 
ny fil på Dropbox.
Post er hentet. Har fått dokumenter fra Nitschke. 
Geir har hentet RF-regi sine SPF-dokumenter.
Venter på Aleksi som skal ha innføring i Norlønn.

\section{Omvisning på FF-kontoret}
Det ble omvisning.

\section{Norlønn -- innføring}
Aleksi har ikke kommet. Tar opplæring i Norlønn og nettsiden
\textbf{fredag 20. februar kl. 11.}.
Hvis Aleksi ikke kan da, lager vi en doodle.

\section{Vitebok / Wiki}
Det finnes ikke en vitebok i SPF. Bør sette en frist for 
å lage en. Mai? Skal ha med alle rutiner, 
og referater fra møter og generalforsamling.

Siden vi ikke er en UiO-forening, får vi neppe wiki gjennom dem. 
Christiane spør Uglebos IT-kontakter om mulighet for å lage wiki. 
Aleksi kan kanskje ta ansvar for den?

\section{Eventuelt}
Ingen saker å melde.

\end{document}
